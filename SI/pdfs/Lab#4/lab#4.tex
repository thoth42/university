%----------------------------------------------------------------------------------------
% PACKAGES AND DOCUMENT CONFIGURATIONS
%----------------------------------------------------------------------------------------

  \documentclass{article}

  \usepackage{hyperref}
  \usepackage{fancyhdr} % Required for custom headers
  \usepackage{lastpage} % Required to determine the last page for the footer
  \usepackage{extramarks} % Required for headers and footers
  \usepackage[usenames,dvipsnames]{color} % Required for custom colors
  \usepackage{graphicx} % Required to insert images
  \usepackage{listings} % Required for insertion of code
  \usepackage{courier} % Required for the courier font
  \usepackage{lipsum} % Used for inserting dummy 'Lorem ipsum' text into the template
  \usepackage{wrapfig}
  \usepackage{color}
  \usepackage{lscape}

  \setlength\parindent{0pt} % Removes all indentation from paragraphs
  \renewcommand{\labelenumi}{\alph{enumi}.} % Make numbering in the itemize environment by letter rather than number (e.g. section 6)

  % Margins
  \topmargin=-0.7in
  \evensidemargin=0.2in
  \oddsidemargin=-0.2in
  \textwidth=7.0in
  \textheight=9.3in
  % \headsep=0.25in

  % \linespread{1.1} % Line spacing

  \definecolor{dkgreen}{rgb}{0,0.6,0}
  \definecolor{gray}{rgb}{0.5,0.5,0.5}
  \definecolor{mauve}{rgb}{0.58,0,0.82}
  \definecolor{greyish}{rgb}{0.96,0.96,0.96}

  \lstset{
    backgroundcolor=\color{greyish},   % choose the background color; you must add \usepackage{color} or \usepackage{xcolor}
    frame=tb,
    numbers=left,                    % where to put the line-numbers; possible values are (none, left, right)
    numbersep=5pt,                   % how far the line-numbers are from the code
    numberstyle=\tiny\color{mygray}, % the style that is used for the line-numbers
    language=Ruby,
    aboveskip=3mm,
    belowskip=3mm,
    showstringspaces=false,
    columns=flexible,
    basicstyle={\footnotesize\ttfamily},
    numbers=none,
    numberstyle=\tiny\color{gray},
    keywordstyle=\color{blue},
    commentstyle=\color{dkgreen},
    stringstyle=\color{mauve},
    breaklines=true,
    breakatwhitespace=true
    tabsize=3
  }
  \lstset{basicstyle=\ttfamily\footnotesize,breaklines=true}
%----------------------------------------------------------------------------------------
% DOCUMENT INFORMATION
%----------------------------------------------------------------------------------------
  \begin{document}
  \begin{titlepage}

%----------------------------------------------------------------------------------------
% TITLE PAGE INFORMATION
%----------------------------------------------------------------------------------------
 \newcommand{\HRule}{\rule{\linewidth}{0.5mm}} % Defines a new command for the horizontal lines, change thickness here
  \begin{center} % Center everything on the page

  %----------------------------------------------------------------------------------------
  % HEADING SECTIONS
  %----------------------------------------------------------------------------------------
  \textsc{\Large Faculty of Computers, Informatics and Microelectronics}\\[0.5cm]
  \textsc{\LARGE Technical University of Moldova}\\[1.2cm] % Name of your university/college
  \vspace{25 mm}
  \textsc{\Large SI}\\[0.5cm] % Major heading such as course name
  %\textsc{\large Laboratory work \#1-3}\\[0.5cm] % Minor heading such as course title
  \textsc{\large Laboratory work \# 4}\\[0.5cm] % Minor heading such as course title

  %----------------------------------------------------------------------------------------
  % TITLE SECTION
  %----------------------------------------------------------------------------------------
  \vspace{10 mm}
  \HRule \\[0.4cm]
  { \LARGE \bfseries Timing attack experiment }\\[0.4cm] % Title of your document
  \HRule \\[1.5cm]

  %----------------------------------------------------------------------------------------
  % AUTHOR SECTION
  %----------------------------------------------------------------------------------------
  \vspace{40mm}

  \begin{minipage}{0.4\textwidth}
  \begin{flushleft} \large
  \emph{Author:}\\
  Petru \textsc{Negrei} % Your name
  \end{flushleft}
  \end{minipage}
  ~
  \begin{minipage}{0.4\textwidth}
  \begin{flushright} \large
  \emph{Supervisor:} \\
  A. \textsc{Railean} % Supervisor's Name
  \end{flushright}
  \end{minipage}\\[4cm]

  \vspace{15 mm}
  % If you don't want a supervisor, uncomment the two lines below and remove the section above
  %\Large \emph{Author:}\\
  %John \textsc{Smith}\\[3cm] % Your name

  %----------------------------------------------------------------------------------------
  % DATE SECTION
  %----------------------------------------------------------------------------------------

  {\large November 2014}\\[3cm] % Date, change the \today to a set date if you want to be precise

  %----------------------------------------------------------------------------------------
  % LOGO SECTION
  %----------------------------------------------------------------------------------------

  %\includegraphics{Logo}\\[1cm] % Include a department/university logo - this will require the graphicx package

  %----------------------------------------------------------------------------------------

  \vfill % Fill the rest of the page with whitespace
  \end{center}
  \end{titlepage}

  % \newpage
  % \tableofcontents
  % \newpage

%----------------------------------------------------------------------------------------
% Introduction
%----------------------------------------------------------------------------------------

  \section{Introduction}

  \subsection{Objective}

  \begin{itemize}
    \item Create a test environment to see whether a timing attack is feasible
    \item  Analyze and document your findings
    \item Compare your results with those of your colleagues
    \item Write a constant-time string comparison function
  \end{itemize}

  \subsection{Definition}

  In cryptography, a timing attack is a side channel attack in which the attacker attempts to compromise a cryptosystem by analyzing the time taken to execute cryptographic algorithms. 
  Every logical operation in a computer takes time to execute, and the time can differ based on the input; with precise measurements of the time for each operation, an attacker can
  work backwards to the input.

%%--------------------------------------------------------------------------
%% Implementation
%%--------------------------------------------------------------------------

  \section{Description}

  The following program analyzes necessary time to compare different strings and time needed for it.

  We check the following cases:

  \begin{itemize}
    \item Compare two strings with == where the characters are the same until late in the string.
    \item Compare two strings with == where the characters differ early in the string.
    \item  Same as 1 but with a ruby implementation of secure\_compare.
    \item Same as 2 but with a ruby implementation of secure\_compare.
  \end{itemize}

  We do 1000 tests of each case. \\

  Due to the fact that the difference is not visible at small sizes, I implemented the following 
  example that uses the 'TrurlAndKlapaucius' repeated 1000 times. \\

  Ruby's benchmark module rounds off numbers quite agressively, so to see anything for the
  == cases, refer to the 'real' measurement. \\

  In the first case, you should observe an order of magnitude or greater difference in using
  just ==, while you get measurements for both secure\_compares that are the same to within
  the margin of error. \\

  In the second case, the == are much closer, but should still be distinguishable. \\

  You'll also notice that secure\_compare is *much* slower and not super consistent,
  especially the pure Ruby one. This is the price you pay for not having a language-level
  secure byte comparison primitive.

  \section{Implementation}

    \begin{lstlisting}
    require "benchmark"
    require "digest"

    def secure_compare(a, b)
      check = a.bytesize ^ b.bytesize
      a.bytes.zip(b.bytes) { |x, y| check |= x ^ y.to_i }
      check == 0
    end


    def compare(base_str)
      early_str = base_str.clone
      early_str[0] = 'z'

      late_str = base_str.clone()
      late_str[late_str.length-1] = '!'

      Benchmark.bmbm do |b|
          b.report("==, early fail")  { 1000.times { base_str == early_str } }
          b.report("==, late fail") {1000.times { base_str == late_str } }
          puts ""
          b.report("Ruby secure_compare, 'early'") { 1000.times { secure_compare(base_str, early_str) } }
          b.report("Ruby secure_compare, 'late'") { 1000.times { secure_compare(base_str, late_str) } }
          puts ""
          b.report("SHA512-then-==, 'early'") { 1000.times { Digest::SHA512.digest(base_str) == Digest::SHA512.digest(early_str) } }
          b.report("SHA512-then-==, 'late'") { 1000.times { Digest::SHA512.digest(base_str) == Digest::SHA512.digest(late_str) } }
        end
    end

    puts ""
    puts "==== Short text ===="
    puts ""
    compare('TrurlAndKlapaucius'*50)

    puts ""
    puts "==== Long text ===="
    puts ""
    compare('TrurlAndKlapaucius'*1000)
    \end{lstlisting}

    \begin{lstlisting}

==== Short text ====
------------------------------------------------------- 
                                   user     system      total        real
==, early fail                 0.000000   0.000000   0.000000 (  0.000115)
==, late fail                  0.000000   0.000000   0.000000 (  0.000158)
Ruby secure_compare, 'early'   0.220000   0.000000   0.220000 (  0.224304)
Ruby secure_compare, 'late'            0.220000   0.000000   0.220000 (  0.232572)
SHA512-then-==, 'early'        0.010000   0.000000   0.010000 (  0.013119)
SHA512-then-==, 'late'          0.020000   0.000000   0.020000 (  0.013085)


==== Long text ====
------------------------------------------------------- 

                                   user     system      total        real
==, early fail                 0.000000   0.000000   0.000000 (  0.000117)
==, late fail                  0.000000   0.000000   0.000000 (  0.001374)
Ruby secure_compare, 'early'   4.360000   0.010000   4.370000 (  4.384387)
Ruby secure_compare, 'late'            4.370000   0.010000   4.380000 (  4.389848)
SHA512-then-==, 'early'            0.190000   0.000000   0.190000 (  0.190479)
SHA512-then-==, 'late'              0.190000   0.000000   0.190000 (  0.191458)

    \end{lstlisting}


 \section{Conclussion}

    After making this laboratory work I learn more about timing attacks, and perform
    different comparision on strings to see the diference between recorded times needed
    to perform the 'comp' operation. Also implemented constant-time string comparison and saw why it is necessary.
    Analyzing the preliminary results, the time received for short strings (usually used for passwords and normal text) is too small and random to draw some conclusion about the 
    used information, but if we use big strings the difference in time is visible for unsave comparision of strings.


\end{document}

