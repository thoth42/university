%----------------------------------------------------------------------------------------
% PACKAGES AND DOCUMENT CONFIGURATIONS
%----------------------------------------------------------------------------------------

  \documentclass{article}

  \usepackage{hyperref}
  \usepackage{fancyhdr} % Required for custom headers
  \usepackage{lastpage} % Required to determine the last page for the footer
  \usepackage{extramarks} % Required for headers and footers
  \usepackage[usenames,dvipsnames]{color} % Required for custom colors
  \usepackage{graphicx} % Required to insert images
  \usepackage{listings} % Required for insertion of code
  \usepackage{courier} % Required for the courier font
  \usepackage{lipsum} % Used for inserting dummy 'Lorem ipsum' text into the template
  \usepackage{wrapfig}
  \usepackage{color}
  \usepackage{lscape}

  \setlength\parindent{0pt} % Removes all indentation from paragraphs
  \renewcommand{\labelenumi}{\alph{enumi}.} % Make numbering in the itemize environment by letter rather than number (e.g. section 6)

  % Margins
  \topmargin=-0.7in
  \evensidemargin=0.2in
  \oddsidemargin=-0.2in
  \textwidth=7.0in
  \textheight=9.3in
  % \headsep=0.25in

  % \linespread{1.1} % Line spacing

  \definecolor{dkgreen}{rgb}{0,0.6,0}
  \definecolor{gray}{rgb}{0.5,0.5,0.5}
  \definecolor{mauve}{rgb}{0.58,0,0.82}
  \definecolor{greyish}{rgb}{0.96,0.96,0.96}

  \lstset{
    backgroundcolor=\color{greyish},   % choose the background color; you must add \usepackage{color} or \usepackage{xcolor}
    frame=tb,
    numbers=left,                    % where to put the line-numbers; possible values are (none, left, right)
    numbersep=5pt,                   % how far the line-numbers are from the code
    numberstyle=\tiny\color{mygray}, % the style that is used for the line-numbers
    language=Ruby,
    aboveskip=3mm,
    belowskip=3mm,
    showstringspaces=false,
    columns=flexible,
    basicstyle={\footnotesize\ttfamily},
    numbers=none,
    numberstyle=\tiny\color{gray},
    keywordstyle=\color{blue},
    commentstyle=\color{dkgreen},
    stringstyle=\color{mauve},
    breaklines=true,
    breakatwhitespace=true
    tabsize=3
  }
%----------------------------------------------------------------------------------------
% DOCUMENT INFORMATION
%----------------------------------------------------------------------------------------
  \begin{document}
  \begin{titlepage}

%----------------------------------------------------------------------------------------
% TITLE PAGE INFORMATION
%----------------------------------------------------------------------------------------
 \newcommand{\HRule}{\rule{\linewidth}{0.5mm}} % Defines a new command for the horizontal lines, change thickness here
  \begin{center} % Center everything on the page

  %----------------------------------------------------------------------------------------
  % HEADING SECTIONS
  %----------------------------------------------------------------------------------------
  \textsc{\Large Faculty of Computers, Informatics and Microelectronics}\\[0.5cm]
  \textsc{\LARGE Technical University of Moldova}\\[1.2cm] % Name of your university/college
  \vspace{25 mm}
  \textsc{\Large SI}\\[0.5cm] % Major heading such as course name
  %\textsc{\large Laboratory work \#1-3}\\[0.5cm] % Minor heading such as course title
  \textsc{\large Laboratory work \# 2}\\[0.5cm] % Minor heading such as course title

  %----------------------------------------------------------------------------------------
  % TITLE SECTION
  %----------------------------------------------------------------------------------------
  \vspace{10 mm}
  \HRule \\[0.4cm]
  { \LARGE \bfseries Encryption: Rot13 }\\[0.4cm] % Title of your document
  \HRule \\[1.5cm]

  %----------------------------------------------------------------------------------------
  % AUTHOR SECTION
  %----------------------------------------------------------------------------------------
  \vspace{40mm}

  \begin{minipage}{0.4\textwidth}
  \begin{flushleft} \large
  \emph{Author:}\\
  Petru \textsc{Negrei} % Your name
  \end{flushleft}
  \end{minipage}
  ~
  \begin{minipage}{0.4\textwidth}
  \begin{flushright} \large
  \emph{Supervisor:} \\
  A. \textsc{Railean} % Supervisor's Name
  \end{flushright}
  \end{minipage}\\[4cm]

  \vspace{15 mm}
  % If you don't want a supervisor, uncomment the two lines below and remove the section above
  %\Large \emph{Author:}\\
  %John \textsc{Smith}\\[3cm] % Your name

  %----------------------------------------------------------------------------------------
  % DATE SECTION
  %----------------------------------------------------------------------------------------

  {\large September 2014}\\[3cm] % Date, change the \today to a set date if you want to be precise

  %----------------------------------------------------------------------------------------
  % LOGO SECTION
  %----------------------------------------------------------------------------------------

  %\includegraphics{Logo}\\[1cm] % Include a department/university logo - this will require the graphicx package

  %----------------------------------------------------------------------------------------

  \vfill % Fill the rest of the page with whitespace
  \end{center}
  \end{titlepage}

  % \newpage
  % \tableofcontents
  % \newpage

%----------------------------------------------------------------------------------------
% Introduction
%----------------------------------------------------------------------------------------

  \section{Introduction}

  \subsection{Objective}

  Understand the rot13 cipher and build a tool that break it.

  \subsection{Definition}

  ROT13 ("rotate by 13 places", sometimes hyphenated ROT-13) is a simple letter substitution
  cipher that replaces a letter with the letter 13 letters after it in the alphabet.
  ROT13 is an example of the Caesar cipher, developed in ancient Rome. \\

 In the basic Latin alphabet, ROT13 is its own inverse; that is, to undo ROT13, the same algorithm is applied, so the same action can be used for encoding and decoding.
 The algorithm provides virtually no cryptographic security, and is often cited as a canonical example of weak encryption.

%%--------------------------------------------------------------------------
%% Implementation
%%--------------------------------------------------------------------------

  \section{Implementation}

  The following program implements two way of breaking the
  rot13  substitution cipher:

  \begin{itemize}
      \item Frequency analysis
      \item Each substitution number
  \end{itemize}

  \subsection{Cypher module}

  The text is received from a existing document.

    \begin{lstlisting}
      def cipher
        str = ""
        File.open("data.txt", "r") do |f|
          f.each_line { |line| str << line }
        end
        rot(str, 13)
      end
    \end{lstlisting}

    Then each letter is replaced by the given number.

    \begin{lstlisting}
      def rot string, num
        origin = "a-zA-Z"
        cipher = [('a'..'z'), ('A'..'Z')].map {|range| range.to_a.rotate(num).join }.join
        string.tr origin, cipher
      end
    \end{lstlisting}

    And the last step is to initiate a dictionary of words.

    \begin{lstlisting}
      def initiate_dictionary
        # /usr/share/dict/words
        File.open("./dictionary.txt") do |file|
          file.each { |line| @words[line.strip] = true }
        end
      end
    \end{lstlisting}

    \newpage

  \subsection{Frequency analysis}

  First we begin by iterating through known most used
  letters in english dictionary and check each for matching
  number of words.

   \begin{lstlisting}
      def rot_num
        chars = ["e", "t", "a", "o", "i", "n"]

        char = chars.detect {|char| valid?(ord_num(char))}
        ord_num char
      end
   \end{lstlisting}

   \begin{lstlisting}
      def valid? num
        rot(cipher, num).split.select{ |word| @words[word.downcase] }.count  > 1
      end

      def ord_num char
        most_used_char.ord - char.ord
      end

      def most_used_char
        @msch ||= cipher.scan(/\w/)
                                         .inject(Hash.new(0)) { |h, c| h[c] += 1; h }
                                         .max_by { |_,v| v }.first
      end
   \end{lstlisting}

   and at the end we output the original string.

   \begin{lstlisting}
      def original
        rot cipher, rot_num
      end
   \end{lstlisting}

  \subsection{Step analysis}

  This implementation is easier to apply, we iterate through each
  possible number of permutation of alphabet (26) and check which
  has the most valid words, and then apply the algorithm.

   \begin{lstlisting}
    def check_rot
        (0..26). map do |num|
          cipher.split.select{ |word| @words[rot(word, num).downcase] }.count
        end
      end

      def rot_num
         26 - check_rot.each_with_index.max[1]
       end

      def show_original
        rot cipher, rot_num
      end
   \end{lstlisting}

 \section{Conclussion}

    After making this laboratory work I learn about different methods to break the ROT13 (and caesar ciphers) substition cypher
    and it is easy to observe that algorithm provide almost no cryptographic security. Some of the possible aproaches to
    improve the security are:

    \begin{itemize}
        \item \textit{The Vignere cipher} -  to use different shifts at different positions in the message. (additional key)
        \item \textit{One-time pads}  - where the previous additional key used is infinitely long or a completely random key.
        \item \textit{Multi-character alphabets} - treate larger chunks of text as the characters of our message.
    \end{itemize}

\end{document}

