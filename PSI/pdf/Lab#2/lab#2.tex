%----------------------------------------------------------------------------------------
% PACKAGES AND DOCUMENT CONFIGURATIONS
%----------------------------------------------------------------------------------------

  \documentclass{article}

  \usepackage{hyperref}
  \usepackage{fancyhdr} % Required for custom headers
  \usepackage{lastpage} % Required to determine the last page for the footer
  \usepackage{extramarks} % Required for headers and footers
  \usepackage[usenames,dvipsnames]{color} % Required for custom colors
  \usepackage{graphicx} % Required to insert images
  \usepackage{listings} % Required for insertion of code
  \usepackage{courier} % Required for the courier font
  \usepackage{lipsum} % Used for inserting dummy 'Lorem ipsum' text into the template
  \usepackage{wrapfig}
  \usepackage{color}
  \usepackage{lscape}

  \setlength\parindent{0pt} % Removes all indentation from paragraphs
  \renewcommand{\labelenumi}{\alph{enumi}.} % Make numbering in the itemize environment by letter rather than number (e.g. section 6)

  % Margins
  \topmargin=-0.7in
  \evensidemargin=0.2in
  \oddsidemargin=-0.2in
  \textwidth=7.0in
  \textheight=9.3in
  % \headsep=0.25in

  % \linespread{1.1} % Line spacing

  \definecolor{dkgreen}{rgb}{0,0.6,0}
  \definecolor{gray}{rgb}{0.5,0.5,0.5}
  \definecolor{mauve}{rgb}{0.58,0,0.82}
  \definecolor{greyish}{rgb}{0.96,0.96,0.96}

  \lstset{
    backgroundcolor=\color{greyish},   % choose the background color; you must add \usepackage{color} or \usepackage{xcolor}
    frame=tb,
    numbers=left,                    % where to put the line-numbers; possible values are (none, left, right)
    numbersep=5pt,                   % how far the line-numbers are from the code
    numberstyle=\tiny\color{mygray}, % the style that is used for the line-numbers
    language=Ruby,
    aboveskip=3mm,
    belowskip=3mm,
    showstringspaces=false,
    columns=flexible,
    basicstyle={\footnotesize\ttfamily},
    numbers=none,
    numberstyle=\tiny\color{gray},
    keywordstyle=\color{blue},
    commentstyle=\color{dkgreen},
    stringstyle=\color{mauve},
    breaklines=true,
    breakatwhitespace=true
    tabsize=3
  }
%----------------------------------------------------------------------------------------
% DOCUMENT INFORMATION
%----------------------------------------------------------------------------------------
  \begin{document}
  \begin{titlepage}

%----------------------------------------------------------------------------------------
% TITLE PAGE INFORMATION
%----------------------------------------------------------------------------------------
 \newcommand{\HRule}{\rule{\linewidth}{0.5mm}} % Defines a new command for the horizontal lines, change thickness here
  \begin{center} % Center everything on the page

  %----------------------------------------------------------------------------------------
  % HEADING SECTIONS
  %----------------------------------------------------------------------------------------
  \textsc{\Large Faculty of Computers, Informatics and Microelectronics}\\[0.5cm]
  \textsc{\LARGE Technical University of Moldova}\\[1.2cm] % Name of your university/college
  \vspace{25 mm}
  \textsc{\Large PSI}\\[0.5cm] % Major heading such as course name
  %\textsc{\large Laboratory work \#1-3}\\[0.5cm] % Minor heading such as course title
  \textsc{\large Laboratory work \# 2}\\[0.5cm] % Minor heading such as course title

  %----------------------------------------------------------------------------------------
  % TITLE SECTION
  %----------------------------------------------------------------------------------------
  \vspace{10 mm}
  \HRule \\[0.4cm]
  { \LARGE \bfseries Registration number verification }\\[0.4cm] % Title of your document
  \HRule \\[1.5cm]

  %----------------------------------------------------------------------------------------
  % AUTHOR SECTION
  %----------------------------------------------------------------------------------------
  \vspace{40mm}

  \begin{minipage}{0.4\textwidth}
  \begin{flushleft} \large
  \emph{Author:}\\
  Petru \textsc{Negrei} % Your name
  \end{flushleft}
  \end{minipage}
  ~
  \begin{minipage}{0.4\textwidth}
  \begin{flushright} \large
  \emph{Supervisor:} \\
  A. \textsc{Railean} % Supervisor's Name
  \end{flushright}
  \end{minipage}\\[4cm]

  \vspace{15 mm}
  % If you don't want a supervisor, uncomment the two lines below and remove the section above
  %\Large \emph{Author:}\\
  %John \textsc{Smith}\\[3cm] % Your name

  %----------------------------------------------------------------------------------------
  % DATE SECTION
  %----------------------------------------------------------------------------------------

  {\large November 2014}\\[3cm] % Date, change the \today to a set date if you want to be precise

  %----------------------------------------------------------------------------------------
  % LOGO SECTION
  %----------------------------------------------------------------------------------------

  %\includegraphics{Logo}\\[1cm] % Include a department/university logo - this will require the graphicx package

  %----------------------------------------------------------------------------------------

  \vfill % Fill the rest of the page with whitespace
  \end{center}
  \end{titlepage}

  % \newpage
  % \tableofcontents
  % \newpage

%----------------------------------------------------------------------------------------
% Introduction
%----------------------------------------------------------------------------------------

  \section{Introduction}

  \subsection{Objective}

  \begin{itemize}
    \item Design and develop a registration number verification mechanism
    \item That relies on public key cryptography
  \end{itemize}

  \subsection{Overview}

  Registration numbers - nobody likes them, especially when they are split into several chunks that have to 
  go to different edit widgets (so you must perform more than one copy/paste operation from the keygen.
  Nevertheless, they are the most common method of verifying the authenticity of the installed application, 
  and most commercial software applies such checks.

%%--------------------------------------------------------------------------
%% Implementation
%%--------------------------------------------------------------------------

  \section{Implementation}

  \begin{itemize}
      \item Frequency analysis
      \item Each substitution number
  \end{itemize}

  \subsection{Cypher module}

  Below there are external ruby libraries used in this program.

  \begin{itemize}
    \item \textit{openssl} - OpenSSL provides general purpose cryptography which includes RSA.
    \item \textit{base64} - The Base64 module provides for the encoding and decoding of binary data using a Base64 representation.
    \item \textit{securerandom} - Generate a random number-string 
  \end{itemize}

  \begin{itemize}
    \item \textit{SECRET} - some predefined secret string.
    \item \textit{PUBLIC\_KEY\_LOC} - public key generate by RSA.
    \item \textit{KEY\_LOC} - encrypted string with the private key.
  \end{itemize}

  \begin{lstlisting}
    require 'openssl'
    require 'base64'
    require 'securerandom'

    SECRET = "s3cr3t"
    PUBLIC_KEY_LOC = "public_key.pub"
    KEY_LOC = "key.txt"
  \end{lstlisting}

  Generator steps:

  \begin{itemize}
    \item First we generate a random string with a secret string.
    \item Then we encrypt that key using the way of public key cryptography.
    \item Then encode the key in base64.
    \item Save the encoded key in a file, later to be read.
  \end{itemize}

  \begin{lstlisting}
  class Generator

    def initialize
      @rsa = OpenSSL::PKey::RSA.new 2048
      save_public_key
      save_serial
    end

    private

    #'O4KLIVc-5u8C11E-s3cr3t-S9NSu1A-s4qWzcI-lUI9k_0-RV9ETlY
    # "P0X2lPU3OQAsQMIRUaocU3bMn8VAv6cK7xcs3cr3t7rPd4_Y"
    def generate_key
      tmp = (0..5).map { || generate_numbers } << SECRET 
      tmp.shuffle.join #"-"
    end

    def generate_numbers
      SecureRandom.urlsafe_base64(5)
    end

    def save_public_key
      File.open(PUBLIC_KEY_LOC, 'w+') { |file| file.write @rsa.public_key.to_pem}
    end

      #encrypt with the private RSA key
      def  encrypted_key
        @rsa.private_encrypt generate_key
      end

      def encoded_key
        Base64.encode64 encrypted_key
      end

      def save_serial
        File.open(KEY_LOC, 'w+') { |file| file.write encoded_key }
      end
  end
  \end{lstlisting}

   \begin{lstlisting}
    module Verifier

      class Checker

        def initialize key
          @key = key
          @rsa = OpenSSL::PKey::RSA.new File.read(PUBLIC_KEY_LOC)
        end

        def result
          !(decrypted_key =~ /"#{SECRET}"/)
        end

        private

        def decoded_key
          Base64.decode64 @key
        end

        def decrypted_key 
          @rsa.public_decrypt decoded_key
        end

      end

        def self.verify key
          Checker.new(key).result
        end
    end

    \end{lstlisting}

    Verifier steps:

    \begin{itemize}
      \item Reverse the base64 encoding.
      \item Decrypt the key using public key.
      \item Check the number and return a boolean value.
    \end{itemize}

    \begin{lstlisting}
    Generator.new
    key = File.open(KEY_LOC, "rb").read
    p Verifier.verify key 
    \end{lstlisting}

    \begin{lstlisting}
    # public_key.pub
    -----BEGIN PUBLIC KEY-----
    MIIBIjANBgkqhkiG9w0BAQEFAAOCAQ8AMIIBCgKCAQEAxdoL7Xt2IrekMgQ8m2IF
    IM/DnfsNvURqRL/Meds+acd5Uw9qyTZJ0AdDwNPJRP7+iRxeX9O/QQ376II/F342
    /gJVZGNUaVfTAEvkfAI8uX6RT1UyQOF6FCH5DwO1h6iQqPKlspQtqDqoZedKzSS5
    kJJ+yAh17xgM53YysgQAXoNO90/HsMQ1+jr7oIKnD84yBhYxfCGMTXNDYGMM3nXa
    vyx5nmrlHk+W0TzSBawpFB1SMkOcHNoZvTJQ+TWcHj+pyzO9noFgrUq/KVuVDS9d
    K6uekbs9t7gIRBaRcBL5wWnf/c5L2jlye9l3OZKs0mbVYRqnyt0dr2zi+JuJgkB5
    YwIDAQAB
    -----END PUBLIC KEY-----
    \end{lstlisting}

    \begin{lstlisting}
    # key.txt
    xM7zDVYX2s/O7PTIMnJXbw62WAxLG7IAf60rB5dsNdJnwWyjk0UayIiOjZVJ
    Ruc7hjO7g1koC+nDYFD72ArmFjOa1l1/AUQ9acLSl3/SJa7Wa5MhHRqi7ySZ
    PcYbgNtgv/A7g/m5pPgtwFmXZy4io2IfbGbFZYssxhUZevuXRDO0JV3GaJPU
    FvTI5u3JtFg6mrSZF1lZZdBhMnOKdpJFqBtkeGveXJaW7GCxHuVPYrOwqCb+
    zq+aISwkuzA3p+K7XsoKGFMjPaZh1Pjwm4BkddNi1Jw4vvRZxWl/ZaJnMxYe
    s7awg4qWayft8OYgdNx16fSpX2FVrqcQ5/jcUcVK3g==
    \end{lstlisting}

    \begin{lstlisting}
    # result 
    true
    \end{lstlisting}

 \section{Conclussion}

    After making this laboratory work I learn more about public key cryptography and how
    it can be used in the case of hiding information about registration numbers or 
    serial number. Also implemented a checker for verification of validity of the serial number, 
    the method used here is far from reality of generating and verificiation of serial numbers 
    but it serve the purpose to show the mechanism of publick key cryptography. 

\end{document}

